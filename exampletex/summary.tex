\newpage
{\let\cleardoublepage\relax \chapter{Podsumowanie}}
\label{cha:summary}

\section{Wnioski}

Celem pracy było wytworzenie projektu aplikacji internetowej wykorzystującej fiszki w celu nauki języka obcego. Projekt został zrealizowany z użyciem najnowszych rozwiązań technologicznych w dziedzinie aplikacji internetowych takich jak ASP.NET MVC, C\#, SASS, Scala.js.

Proces nauki został wykonany na podstawie samodzielnie stworzonych wzorów opartych o krzywą zapominania i system powtórzeń, których parametry były dostosowywane na podstawie przeprowadzonych symulacji. Dzięki tym rozwiązaniom proces nauki z wykorzystaniem tego projektu będzie efektywniejszy.

Aplikacja została stworzona zgodnie ze wzorcem MVC. Dzięki temu poszczególne warstwy aplikacji (przechowywania, przetwarzania, wyświetlania danych) mogą być oddzielnie modyfikowane bez wpływu na inne. Rozwiązanie to wpływa korzystnie na czytelność kodu programu.

\subsection{Scala.js}

Integracja z językiem Scala.js w projekcie ASP.NET MVC została zakończona z powodzeniem. Jest to w pełni możliwe do wykonania pomimo braku jakiegoś narzędzia integrującego Visual Studio z tym językiem. 

Rozwiązanie to działało bardzo dobrze. Pisanie w tym języku było bardzo przyjemne od strony programistycznej, dzięki możliwości skupienia się na kodzie i nie zwracaniu uwagi na dziwne zachowania języka, które występowały w kodzie JavaScript.
Niestety to rozwiązanie ma poważną wadę. Z racji tego jak działa proces kompilacji Scali.js niemożliwym jest aby utworzyć kilka plików z kodem JavaScript. W dzisiejszym świecie działaniem pożądanym jest zmniejszenie rozmiaru i ilości zapytań HTTP jak to tylko możliwe. Niemożność rozdzielenia kodu na kilka modułów i wczytanie tylko jednej wymaganej części powoduje zwiększenie ilości pobieranych danych.
Co więcej, należy mieć na uwadze fakt, iż to rozwiązanie jest niepopularne wśród programistów. W związku z tym, wszelkie napotkane problemy będą prawdopodobnie rozwiązywane w pojedynkę bez pomocy zewnętrznych źródeł.

\section{Możliwe ulepszenia}

Interfejs aplikacji jest napisany w języku angielskim. Pomimo wykorzystania jedynie języka obcego w trakcie nauki, interfejs mógłby być tłumaczony na język ojczysty danego użytkownika. Ułatwiłoby to posługiwanie się aplikacją dla osób nieznających języka angielskiego.

Aktualnie dla opisu danej fiszki są używane obrazki oraz tłumaczenia, które wykorzystywane są w trakcie procesu nauczania. Należałoby rozważyć dodanie nowych form opisu fiszki, które pomogłyby w nauce. Przykładem nowej funkcjonalności mogłoby być dodanie lektora, który wypowiadałby dane tłumaczenia podczas nauki. Taki dodatek mógłby pozytywnie wpłynąć na efektywność zapamiętywania.